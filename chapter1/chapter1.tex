\chapter{Введение}

Данная лабораторная работа рассчитана на четыре занятия. Задание выполняется бригадой численностью от 5 до 8 человек (как правило, из учебной группы формируется две бригады). В каждой бригаде по итогам голосования назначается руководитель проекта, который отвечает за распределение обязанностей между студентами в бригаде. Руководителя проекта можно переизбирать в процессе выполнения лабораторных работ, для чего нужно в письменной форме уведомить преподавателя.

\textit {Цель лабораторной работы} --- закрепить навыки программирования на языке Python, освоить систему GitHub для контроля версий ПО с открытым программным кодом и поучить навыки коллективной разработки программного обеспечения на примере разработки программы имитационного моделирования радиосистемы передачи данных.
 
 
\section{Задание}

Составить при помощи ЭВМ имитационную модель цифровой радиосистемы системы передачи информации, которая должна включать в себя следующие составные части:
\begin{enumerate}
\item	Источник информации.

\item	Модуль сжатия исходной информации.

\item	Модуль помехоустойчивого кодирования.

\item	Модуль модуляции.

\item	Модуль наложения шума и помех.

\item	Модуль демодуляции.

\item	Модуль декодирования.

\item	Модуль восстановления исходной информации.

\item	Модуль статистики.

\item	Графический интерфейс.

\item	Сопроводительная документация.
\end{enumerate}
\section{Дополнительные требования к заданию}

Каждый модуль программы должен быть реализован в виде отдельной процедуры или функции к которому прилагаются тестовые примеры для проверки работоспособности. В графическом интерфейсе программы должна быть заложена возможность визуального контроля сигналов или информационной последовательности на каждом этапе выполнения программы (построения графиков по средствам меню или кнопок на форме).

 В качестве языка программирования разрешено использовать только Python и C++. Для контроля версий ПО обязательно использовать систему GitHab. Для составления сопроводительной документации обязательно использовать текстовый редактор LaTeX.

\section{Требования к составным частям имитационной модели}

\begin{enumerate}
\item	Источник информации должен генерировать случайную последовательность, состоящую из 512 двоичных символов.

\item	Модуль сжатия информации должен устранять избыточность из исходной последовательности и вычислять коэффициент сжатия.

\item	В модуле помехоустойчивого кодирования требуется реализовать несколько алгоритмов в соответствии с вариантом задания. Выбор алгоритма помехоустойчивого кодирования должен осуществляться пользователем по средствам графического интерфейса.

\item	Методы модуляции и их параметры выбираются в соответствии с вариантом задания. Все параметры должны отображаться при помощи графического интерфейса, кроме того, у пользователя должна быть возможность оперативно изменять параметры модуляции.  При использовании нескольких методов модуляции требуется выбор конкретной модуляции по средствам меню или графического интерфейса.

\item	Требования к шумам и помехам указаны в варианте задания. Необходимо предусмотреть возможность оперативного изменения параметров шума или помехи по средствам графического интерфейса.

\item	Модуль демодуляции должен соответствовать  используемой модуляции.

\item	Процедура декодирования должна соответствовать процедуре кодирования.

\item	Процедура восстановления исходной информации должна соответствовать процедуре сжатия.

\item	Модуль статистики должен вычислять вероятность ошибки в символе в зависимости от уровня шума и наличия помех на основе \textit{n} выборок исходной информационной последовательности, количество которых должно задаваться при помощи графического интерфейса.
\end{enumerate}

\section{Требования к сопроводительной документации}

Сопроводительная документация должна быть набрана в системе компьютерной верстки \LaTeX{} и поставляться в виде исходных \verb\.tex\ файлов и готового файла формата \verb\pdf\. В документации должно быть:

\begin{enumerate}

\item Описание программы

\item Описание отдельных модулей, классов и функций, входящих в ее состав

\begin{enumerate}

\item Назначение класса, модуля, функции

\item Входные и выходные данные

\item Параметры конструктора

\item Пример использования в коде

\end{enumerate}

\item Описание установки необходимых для сборки файлов и библиотек

\item Процедура сборки и запуска программы

\item Скриншоты работы программы

\item Любая другая, необходимая по вашему мнению, информация

\end{enumerate}

Документация должная располагаться в папке \verb\doc\ в корне папки вашего проекта. 


\section{Требования к оформлению кода}

Весь код должен быть набран в едином стиле, с едиными размерами отступов, имена переменных должны быть понятны и осмысленны. Неочевидные участки дома и все интерфейсы закомментированы. 


Для оформления кода используйте общепринятые стандарты кодирования. PEP8 - для Python или \textit{Google C++ Style Guide, Linux kernel coding style, Mozilla C++ Portability Guide, WebKit Coding Style Guidelines.} в случае С++ кода. Допускается в определенных случаях не строгое следования стандарту, однако в приделах бригады все участники должны придерживаться одного набора правил.

