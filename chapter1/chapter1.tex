\chapter{Название главы}

\begin{itemize}
	\item пункт 1
	\item пункт 2
	\item пункт 3
	\begin{itemize}
		\item подпункт 1
		\item подпункт 1
		\item подпункт 1
	\end{itemize}
	\item пункт 4
\end{itemize}

\section{Секция}
\subsection{Подсекция}
\subsubsection{ПодПодсекция}

\begin{figure}[H]
\begin{center}
\includegraphics[scale=0.6]{./image/1.jpg}
\end{center}
\caption{Картина}
\end{figure}

\textit{курсив}, \textbf{жирный}\footnote{сноска} , \verb\моноширный\, 

Lorem ipsum dolor sit amet, consectetur adipiscing elit. Cras cursus neque vel libero iaculis vehicula. 
продолжаем текст на этой же строке

Что бы сделать новый абзац надо дважды поставить символ переноса строки(Enter) Lorem ipsum dolor sit amet, consectetur adipiscing elit. 

Вот так, опять новый абзац Lorem ipsum dolor sit amet, cursus consectetur venenatis. Fusce blandit orci a neque ultrices, id vestibulum risus scelerisque. Sed at magna odio.

Если надо вставить отдельно какой то код то делаем так:
\begin{verbatim}
class Mid{
/*
 * Какой то комментарий
 */
public:
	// Константы при работе с группами портов
	static const unsigned int GROUP_ALL = 0x0;
	static const unsigned int GROUP_A = 0x1;
	static const unsigned int GROUP_B = 0x2;
	static const unsigned int GROUP_C = 0x3;
	static const unsigned int GROUP_D1 = 0x4;
\end{verbatim}

Теперь формулы, если надо вставить формулу с нумерацией, то тогда так:
\begin{equation}\label{Example} %это название ссылки на эту формулу
k^{10^{-6}} = \sqrt{\frac{Q_{i}}{10^{12}}}
\end{equation}
или мы хотим в тексте вставить формулу без номера $ \forall x \in \Re ~\exists y, ~y_{i} \neq \int \sum^{0}_{i} $

Да, знак \verb\~\ означает неразрывный пробел, а если надо ввести какой то служебный символ латеха, чтобы он не обрабатывался, ставится обратный слеш, например \%

Если надо сослаться на формулу \ref{Example}

И на последок, что бы автоматически сгенерировалось оглавление и ссылки, нужно дважды скомпилировать документ, через F1 или кнопку PDFLaTeX в Texmaker