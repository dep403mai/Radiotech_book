\chapter{Варианты заданий}
Во всех вариантах использовать дискретный бинарный источник информации (равновероятное возникновение нулей и единиц). Длина последовательности 512 символов. Модуль сжатия исходной информации используется на усмотрение студентов (использование не обязательно). В канале связи присутствует только аддитивный белый гауссовский шум. Мощность сигнала и шума выбирать таким образом, чтобы отношение сигнал/шум  Остальные требования указаны в таблице.


\begin{center}
\begin{tabular}{|c|p{4cm}|c|c|}
\hline
Номер варианта & Методы кодирования & Кратность исправляемой ошибки & Виды модуляции \\
\hline
1 & код Рида-Саломона и БЧХ-код & 5 & QAM-128 и PM-16\\
\hline
2 & код Хемминга и БЧХ-код & 7 & QAM-256 и QPSK \\
\hline
\end{tabular}
\end{center}

 