\chapter{Теоретические сведения}

\section{Основы программирования на языке \textit{Python}}

\section{Основы работы с системой \textit{GitHub}}

\section{Основы работы с текстовым редактором \LaTeX{}}

\section{Справочные материалы по радиотехнике}

\subsection{Источник информации}

\textbf{Источник информации} --- устройство, передающее информацию по средствам цифровой системы связи. Источник информации может быть аналоговым и дискретным. Выход аналогового источника может иметь любое значение из непрерывного диапазона амплитуд, тогда как выход источника дискретной информации --- значения из конечного множества амплитуд. Источники аналоговой информации преобразуются в источники цифровой информации по средствам квантования и дискретизации.

В лабораторных работах будет использоваться только дискретный, бинарный источник информации с равновероятным возникновением одного из двух событий (выпадение 0 или 1).

\subsection{Каналы связи, шумы, помехи}

\textbf{Канал связи} --- среда распространения, или электромагнитный тракт, соединяющий передатчик и приемник. Каналы связи могут быть воздушные, кабельные, световодные, тропосферные, спутниковые, непрерывные (на входе и выходе канала - непрерывные сигналы), дискретные или цифровые (на входе и выходе канала - дискретные сигналы), непрерывно-дискретные (на входе канала - непрерывные сигналы, а на выходе - дискретные сигналы), дискретно-непрерывные (на входе канала - дискретные сигналы, а на выходе - непрерывные сигналы) и др. В любом канале связи к сигналу добавляется шум или помеха.

Наиболее распространенная модель шума, используемая в теории связи --- аддитивный белый гауссовский шум (АБГШ). Из названия следует, что АБГШ складывается с сигналом в канале связи. Такой шум с точки зрения статистики имеет гауссовское распределение.

\subsection{Помехоустойчивое кодирование}
\textbf{Помехоустойчивое кодирование} вносит избыточность в исходную информационную последовательность. На основе дополнительной (избыточной) информации появляется возможность обнаруживать и исправлять ошибки с определенной кратностью. Кратности исправляемой и определяемой ошибок определяются свойствами конкретного помехоустойчивого кода, в том числе количеством избыточных символов, добавляемых к исходной последовательности.
\subsection{Модуляция}

\subsection{Статистические характеристики системы}